% INTRODUÇÃO-------------------------------------------------------------------

\chapter{INTRODUÇÃO}
\label{chap:introducao}
\textcolor{magenta}
{Antes de começar a escrever você precisará editar o arquivo {\ttfamily capa.tex},  no diretório  {\ttfamily /elementos-pre-textuais}.
Nesse arquivo deverá ser informado nome do autor, título do trabalho, natureza do trabalho, nome do orientador e outras informações necessárias.}

A Introdução é a parte inicial do texto, na qual devem constar o tema e a delimitação do assunto tratado, objetivos da pesquisa e outros elementos necessários para situar o tema do trabalho, tais como: o problema de pesquisa, justificativa, procedimentos metodológicos (classificação inicial) e estrutura do trabalho, tratados de forma sucinta. 

Primeiro Parágrafo: Faça uma contextualização geral de onde seu tema está inserido. 

Segundo Parágrafo: Descreva o que é o seu tema, quando surgiu, qual sua aplicação, qual sua relevância?



\section{Problema de Pesquisa}
\label{sec:problema}
Qual o cenário do problema, quais as principais problemas/dificuldades que vc pretende resolver? 

\section{Objetivos}
Escreva algo como: Os principais objetivos do trabalho são apresentados a seguir.

\subsection{Objetivo Geral}
Qual o objetivo geral?


\subsection{Objetivos Específicos}
Quais os objetivos específicos?

\section{Justificativa}
Por que seu tema de trabalho é importante? Em que contexto? Existem evidências sobre sua aplicação/uso?

Por que sua hipótese é relevante para o contexto do seu problema?

\section{Materiais e Métodos}

De forma bem resumida, escreva os materiais/softwares... que serão utilizados? Quais serão as etapas para desenvolver o projeto?

\section{Organização do Trabalho}
\label{sec:organizacaoTrabalho}

Normalmente ao final da introdução é apresentada, em um ou dois parágrafos curtos, a organização do restante do trabalho acadêmico.
Deve-se dizer o quê será apresentado em cada um dos demais capítulos.