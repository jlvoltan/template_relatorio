% CAPA---------------------------------------------------------------------------------------------------

% PREENCHER AS INFORMAÇÕES GERAIS---------------------------------


% DADOS DO TRABALHO--------------------------------------------------------------------------------------
\titulo{Título do Trabalho}
\autor{NOME DO AUTOR}
\autorcitacao{SOBRENOME, Nome} % Sobrenome em maiúsculo
\local{Rio de Janeiro}
\data{2022}        % Colocar o ano do trabalho

% DADOS DOS ORIENTADORES---------------------------------------------------------------------------------
\orientador{Prof.ª Dra. xxxxxxxxxxxxxl}


% NATUREZA DO TRABALHO----------------------------------------------------------------------------------- 
% *não modificar*
\projeto{Relatório da disciplina Projeto e Análise de Algoritmos}

% TÍTULO ACADÊMICO---------------------------------------------------------------------------------------
% Opções:
% - Licenciado, Bacharel ou Tecnólogo (Se a natureza for Trabalho de Conclusão de Curso)
% - Mestre (Se a natureza for Dissertação)
% - Doutor (Se a natureza for Tese)
% - Mestre ou Doutor (Se a natureza for Projeto de Qualificação)
\tituloAcademico{}

% ÁREA DE CONCENTRAÇÃO E LINHA DE PESQUISA---------------------------------------------------------------
% Se a natureza for Trabalho de Conclusão de Curso, deixe ambos os campos vazios
% Se for programa de Pós-graduação, indique a área de concentração e a linha de pesquisa
\areaconcentracao{}
\linhapesquisa{}

% DADOS DA INSTITUIÇÃO-----------------------------------------------------------------------------------
% Se a natureza for Trabalho de Conclusão de Curso, coloque o nome do curso de graduação em "programa"
\subordinacao{MINISTÉRIO DA DEFESA\\ EXÉRCITO BRASILEIRO \\ DEPARTAMENTO DE CIÊNCIA E TECNOLOGIA}
\instituicao{Instituto Militar de Engenharia}
\programa{Programa de Pós-Graduação em Sistemas e Computação }





% TIPO DE LICENÇA CREATIVE COMMONS

%Licença Creative - Aqui pode ser alterada a forma de licença do trabalho de acordo com a necessidade, levando em consideração o Creative Commons. Verifique a classificação de seu trabalho e use o respectivo comando para inclusão da licença. 

\licencaCC{CC-BY-SA}

% Para usar outro tipo de licença CC, basta comentar o comando acima e descomentar algum outro comando abaixo.
%
%\licencaCC{CC-BY}
%\licencaCC{CC-BY-NC}
%\licencaCC{CC-BY-NC-ND}
%\licencaCC{CC-BY-NC-SA}
%\licencaCC{CC-BY-ND}