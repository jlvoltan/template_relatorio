% ANEXO------------------------------------------------------------------------

\begin{anexosenv}
%\partanexos Tirado na v2.1

% Primeiro anexo---------------------------------------------------------------
\chapter{Nome do anexo}     % edite para alterar o título deste anexo
\label{chap:anexoA}

Lembre-se que a diferença entre apêndice e anexo diz respeito à autoria do texto e/ou material ali colocado.

Caso o material ou texto suplementar ou complementar seja de sua autoria, então ele deverá ser colocado como um apêndice. Porém, caso a autoria seja de terceiros, então o material ou texto deverá ser colocado como anexo.

Caso seja conveniente, podem ser criados outros anexos para o seu trabalho acadêmico. Basta recortar e colar este trecho neste mesmo documento. Lembre-se de alterar o "label"{} do anexo.

Organize seus anexos de modo a que, em cada um deles, haja um único tipo de conteúdo. Isso facilita a leitura e compreensão para o leitor do trabalho. É para ele que você escreve.

% Novo anexo-------------------------------------------------------------------
\chapter{Nome do outro anexo}
\label{chap:anexoB}

conteúdo do outro anexo

\end{anexosenv}
