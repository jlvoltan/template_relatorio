% ---------------------------------------------------------------------------------------------------------------
% TEMPLATE PARA TRABALHO DE CONCLUSÃO DE CURSO



% A ideia de criar este template veio da necessidade de um relatório de uma disciplina.
% Como já havia tido contato com o template utilizado no curso de pós-graduação em ciência de dados da UTFPR, Dois Vizinhos - PR, e o achava de alta qualidade, bem explicado e estruturado, tive a ideia de utilizá-lo como base e montar, através de adaptações um voltado para modelo de relatório de disciplina do Programa de Pós-Graduação em Sistemas e Computação (PGSC) do IME. 

% ESSE TEMPLATE NÃO CONSTITUI UM MODELO DA INSTITUIÇÃO, é apenas um template que resolvi criar através de adaptações com a finalidade de suprir uma necessidade pessoal.

% José Luiz N. Voltan - Aluno PGSC/IME


% Dessa forma, dando crédito a quem merece o crédito, destaco que o template original foi desenvolvido pelas seguintes pessoas:

% Universidade Tecnológica Federal do Paraná - UTFPR
% Curso de Licenciatura em Informática
% Francisco Beltrão - PR
% Customização da classe abnTeX2 (http://www.abntex.net.br/) para as normas da UTFPR
% Autor: Adair Rohling
% Adaptações: Paulo Jr. Varela e Anderson C. Carniel



% utilizei como base a versão  v. 2.2   (01/12/2021) do referido template , ele encontra-se disponível em https://www.overleaf.com/read/vgqbvmnmfppd



%----------------------------------------------------------------------------------------------------------------
% Codificação: UTF-8
% LaTeX:  abnTeX2          
% ---------------------------------------------------------------------------------------------------------------

% CARREGA CLASSE PERSONALIZADA--------------------------------------------------------------------------
\RequirePackage{scrlfile}
\AfterClass{memoir}{\usepackage[a-3b]{pdfx}} %v.3

\documentclass[%twoside,                   % Impressão em frente e verso
    	        oneside,                   % Impressão apenas frente
]{configuracoes/pgsc-abntex2}


% INCLUI ARQUIVOS DE CONFIGURAÇÕES-------------------------------------------------------------------------------
\include{configuracoes/pacotes}
\include{configuracoes/configuracoes-pdf}

% INCLUI ARQUIVOS DO TRABALHO DE CONCLUSÃO DE CURSO (PRÉ-TEXTUAIS, TEXTUAIS, PÓS-TEXTUAIS)-----------------------

% INSERE CAPA E FOLHA DE ROSTO
% CAPA---------------------------------------------------------------------------------------------------

% PREENCHER AS INFORMAÇÕES GERAIS---------------------------------


% DADOS DO TRABALHO--------------------------------------------------------------------------------------
\titulo{Título do Trabalho}
\autor{NOME DO AUTOR}
\autorcitacao{SOBRENOME, Nome} % Sobrenome em maiúsculo
\local{Rio de Janeiro}
\data{2022}        % Colocar o ano do trabalho

% DADOS DOS ORIENTADORES---------------------------------------------------------------------------------
\orientador{Prof.ª Dra. xxxxxxxxxxxxxl}


% NATUREZA DO TRABALHO----------------------------------------------------------------------------------- 
% *não modificar*
\projeto{Relatório da disciplina Projeto e Análise de Algoritmos}

% TÍTULO ACADÊMICO---------------------------------------------------------------------------------------
% Opções:
% - Licenciado, Bacharel ou Tecnólogo (Se a natureza for Trabalho de Conclusão de Curso)
% - Mestre (Se a natureza for Dissertação)
% - Doutor (Se a natureza for Tese)
% - Mestre ou Doutor (Se a natureza for Projeto de Qualificação)
\tituloAcademico{}

% ÁREA DE CONCENTRAÇÃO E LINHA DE PESQUISA---------------------------------------------------------------
% Se a natureza for Trabalho de Conclusão de Curso, deixe ambos os campos vazios
% Se for programa de Pós-graduação, indique a área de concentração e a linha de pesquisa
\areaconcentracao{}
\linhapesquisa{}

% DADOS DA INSTITUIÇÃO-----------------------------------------------------------------------------------
% Se a natureza for Trabalho de Conclusão de Curso, coloque o nome do curso de graduação em "programa"
\subordinacao{MINISTÉRIO DA DEFESA\\ EXÉRCITO BRASILEIRO \\ DEPARTAMENTO DE CIÊNCIA E TECNOLOGIA}
\instituicao{Instituto Militar de Engenharia}
\programa{Programa de Pós-Graduação em Sistemas e Computação }





% TIPO DE LICENÇA CREATIVE COMMONS

%Licença Creative - Aqui pode ser alterada a forma de licença do trabalho de acordo com a necessidade, levando em consideração o Creative Commons. Verifique a classificação de seu trabalho e use o respectivo comando para inclusão da licença. 

\licencaCC{CC-BY-SA}

% Para usar outro tipo de licença CC, basta comentar o comando acima e descomentar algum outro comando abaixo.
%
%\licencaCC{CC-BY}
%\licencaCC{CC-BY-NC}
%\licencaCC{CC-BY-NC-ND}
%\licencaCC{CC-BY-NC-SA}
%\licencaCC{CC-BY-ND}
\include{estrutura/pre-textuais/folha-rosto}

\begin{document}

\pretextual
\imprimircapa                                               	          
\imprimirfolhaderosto{ }   
           
% INSERE ELEMENTOS PRÉ-TEXTUAIS
%\include{estrutura/pre-textuais/dedicatoria}          			   % Dedicatória
%\include{estrutura/pre-textuais/agradecimentos}        			   % Agradecimentos
%\include{estrutura/pre-textuais/epigrafe}              			   % Epígrafe
\include{estrutura/pre-textuais/resumo}             			   % Resumo em Português
%\include{estrutura/pre-textuais/abstract}             		           % Resumo em Inglês
%\include{estrutura/pre-textuais/listas/listas-ilustracoes/lista-figuras}   % Lista de Figuras
%\include{estrutura/pre-textuais/listas/listas-ilustracoes/lista-quadros}   % Lista de Quadros
%\include{estrutura/pre-textuais/listas/lista-tabelas}         		   % Lista de Tabelas
%\include{estrutura/pre-textuais/listas/lista-siglas}          		   % Lista de Abreviaturas e Siglas
%\include{estrutura/pre-textuais/listas/lista-simbolos}        		   % Lista de Símbolos
%\include{estrutura/pre-textuais/listas/listas-diversas/lista-algoritmos}   % Lista de Algoritmos
\include{estrutura/pre-textuais/sumario}               			   % Sumário

\textual
% INSERE ELEMENTOS TEXTUAIS
% INTRODUÇÃO-------------------------------------------------------------------

\chapter{INTRODUÇÃO}
\label{chap:introducao}
\textcolor{magenta}
{Antes de começar a escrever você precisará editar o arquivo {\ttfamily capa.tex},  no diretório  {\ttfamily /elementos-pre-textuais}.
Nesse arquivo deverá ser informado nome do autor, título do trabalho, natureza do trabalho, nome do orientador e outras informações necessárias.}

A Introdução é a parte inicial do texto, na qual devem constar o tema e a delimitação do assunto tratado, objetivos da pesquisa e outros elementos necessários para situar o tema do trabalho, tais como: o problema de pesquisa, justificativa, procedimentos metodológicos (classificação inicial) e estrutura do trabalho, tratados de forma sucinta. 

Primeiro Parágrafo: Faça uma contextualização geral de onde seu tema está inserido. 

Segundo Parágrafo: Descreva o que é o seu tema, quando surgiu, qual sua aplicação, qual sua relevância?



\section{Problema de Pesquisa}
\label{sec:problema}
Qual o cenário do problema, quais as principais problemas/dificuldades que vc pretende resolver? 

\section{Objetivos}
Escreva algo como: Os principais objetivos do trabalho são apresentados a seguir.

\subsection{Objetivo Geral}
Qual o objetivo geral?


\subsection{Objetivos Específicos}
Quais os objetivos específicos?

\section{Justificativa}
Por que seu tema de trabalho é importante? Em que contexto? Existem evidências sobre sua aplicação/uso?

Por que sua hipótese é relevante para o contexto do seu problema?

\section{Materiais e Métodos}

De forma bem resumida, escreva os materiais/softwares... que serão utilizados? Quais serão as etapas para desenvolver o projeto?

\section{Organização do Trabalho}
\label{sec:organizacaoTrabalho}

Normalmente ao final da introdução é apresentada, em um ou dois parágrafos curtos, a organização do restante do trabalho acadêmico.
Deve-se dizer o quê será apresentado em cada um dos demais capítulos.                		           % Introdução
% REVISÃO DE LITERATURA--------------------------------------------------------
%%%% CAPÍTULO 2 - REVISÃO DA LITERATURA (OU REVISÃO BIBLIOGRÁFICA, ESTADO DA ARTE, ESTADO DO CONHECIMENTO)
%%
%% O autor deve registrar seu conhecimento sobre a
%% literatura básica do assunto, discutindo e 
%% comentando a informação já publicada. A revisão deve
%% ser apresentada, preferencialmente, em ordem
%% cronológica e por blocos de assunto, procurando
%% mostrar a evolução do tema.

\chapter{REVISÃO DE LITERATURA}
\label{chap:fundamentacaoTeorica}

É uma boa prática iniciar cada novo capítulo com um breve texto introdutório (tipicamente, dois ou três parágrafos) que deve deixar claro o quê será discutido no capítulo, bem como a organização do capítulo.


         % Revisão de Literatura
%%%% CAPÍTULO 3 - MATERIAIS E MÉTODOS (PODE SER OUTRO TÍTULO DE ACORDO COM O TRABALHO REALIZADO)

\chapter{MATERIAIS E MÉTODOS}
\label{chap:materiais_metodos}
Cada capítulo deve conter uma pequena introdução (tipicamente, um ou dois parágrafos) que deve deixar claro o objetivo e o que será discutido no capítulo, bem como a organização do capítulo.

\section{Materiais}
\label{sec:materiais}

Inserir seu texto aqui...

\section{Métodos}
\label{sec:metodos}

Inserir seu texto aqui...                   % Metodologia
\include{estrutura/textuais/desenvolvimento/resultados}                    % Resultados

%%%%%%%%%%%%%%%%%%%%%%%%%%%%%%%%%%%%%%%%%%%%%%%%%%%%%%%%%%%%%%%%%%%%%
% Lembre de comentar o item a seguir
\include{estrutura/textuais/desenvolvimento/orientacoes}                   % Capítulo com Orientações de uso do Template
%%%%%%%%%%%%%%%%%%%%%%%%%%%%%%%%%%%%%%%%%%%%%%%%%%%%%%%%%%%%%%%%%%%%%



% CONCLUSÃO--------------------------------------------------------------------

\chapter{CONCLUSÃO}
\label{chap:conclusao}

É importante fazer uma análise crítica do trabalho, destacando os possíveis resultados e contribuições para a área de pesquisa.

\section{Limitações}

\section{Trabalhos Futuros}

Também deve indicar, se possível e/ou conveniente, como o trabalho pode ser estendido ou aprimorado.

\section{Considerações Finais}

Encerramento do trabalho acadêmico.
                 			   % Conclusão

\postextual
% INSERE ELEMENTOS PÓS-TEXTUAIS
\include{estrutura/pos-textuais/referencias}           			   % Referências


\end{document}
